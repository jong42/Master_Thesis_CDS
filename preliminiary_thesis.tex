\documentclass{article}

% packages and configurations
\usepackage{animate} % needed for animations and videos
\usepackage[utf8]{inputenc}	% für Umlaute ect.
\usepackage{fancyhdr} % für header
\usepackage{lastpage} % für footer
\usepackage{extramarks} % für header und footer
\usepackage{amsthm} % math stuff
\usepackage{amsmath} % math stuff
\usepackage{amssymb} % math stuff
\usepackage{color}
\usepackage{listings} % code listings
\usepackage{graphicx} % für graphics
\usepackage{color}
\usepackage{tikz}
\usepackage[absolute,overlay]{textpos} %to translate graphics through space
\usepackage{soul}
\usepackage{hyperref}
\usepackage{xcolor}
\usepackage{textpos}
\usepackage{caption}
\usepackage{parcolumns}
\usepackage{enumerate}
\usepackage[ngerman]{babel} % Umlaute
\usepackage[T1]{fontenc}    % this is needed for correct output of umlauts in pd
\usepackage[section]{placeins} %forces placeins to stay in section
\usepackage{datetime} % custom dates
\usepackage{afterpage}
\usepackage[section]{placeins}


\title{Building an interface between probabilistic programming languages and lumen}
\author{Jonas Aaron Gütter  \\
	Friedrich Schiller Universität Jena  \\
    Matrikelnr 152127 \\
    Prof.Dr. Joachim Giesen \\
    M. Sc. Phillip Lucas
	}



\begin{document}

\maketitle

\section{read related material}
        \subsection{understand what Probabilistic Programming Languages (PPLs) are}
        \subsection{understand main idea of Lumen and what we want to do with it}
        \subsection{understand and formulate “why \& what”}
\section{choose a PPL}
        \subsection{work out requirements to chose PPL}
        \subsection{work out preferred (but not necessarily required) features}
        \subsection{chose a PPL based on these requirements and preferences}
\section{get started with PPL}
        \subsection{play around, learn how to use it, what it can do, etc}
        \subsection{also confirm identified 'pain point', i.e. understand and fomulate what problem you are trying to solve, why this is relevant and outline how you plan to solve that pain point}
\section{design a wrapper of chose PPL for Backend of Lumen}
\section{give presentation about work so far, its justification, relevance, verification ideas, etc etc}
\section{implement and test wrapper}
\section{evaluate implementation in terms of goals set in beginning}

\section{interessante PPLs}

    stan for python: https://pystan.readthedocs.io/en/latest/ \newline
    pymc3: https://docs.pymc.io/notebooks/getting\_started.html\#Case-study-2:-Coal-mining-disasters \newline
    edward: http://edwardlib.org/getting-started \newline
    pyro: http://pyro.ai/



\end{document}